\section{Legalität}

Doch selbst wenn Bitcoin wie eine Währung verwendet werden kann, ist noch immer nicht klar ob es gesetzlich wie eine Währung oder eine Ware behandelt werden soll.
Denn Bitcoin weist sehr änliche Eigenschaften wie Gold auf (Knappheit und Fälschunssicherheit), welches allerdings nicht als Währung gilt.
\Cf[32f.]{schilling}
Derzeit befindet sich Bitcoin in einem legalen Graubereich.
Dadurch dass viele Regierungen und Strafverfolgungsbehörden längere Zeit nicht darauf aufmerksam geworden sind wurden Bitcoins öfters für Delikte wie Geldwäsche \cf[33]{schilling} oder Drogenhandel \cf[46]{peicu} verwendet, was durch die Pseudonymität von Bitcoin ebenfalls begünstigt wird.

Besonders der Fall von \emph{Silk Road}, einer Ebay-artigen Website, auf welcher unter anderem Drogen und Waffen gegen Bitcoins angeboten wurde, erlangte 2013 Medienaufmerksamkeit, als das FBI eine Razzia durchführte, die Website vom Netz nahm und 26.000 Bitcoins (damals 4 millionen US-Dollar wert) beschlagnamte.
Obwohl die Aktivität auf Silk Road nur einen kleinen Teil der Transaktionen im Bitcoin-Netzwerk ausmachte, erzeugten die Medien den Eindruck es wäre der Zweck von Bitcoin anonyme Zahlungen zu tätigen, auch aufgrund der Tatsache, dass Bitcoin damit zum ersten Mal in den Massenmedien war.
\Cf[46f.]{peicu}

Doch selbst wenn man legalen Geschäften mit Bitcoin nachgehen möchte, bewegt man sich oft in einem Graubereich.
Es ist unklar, ob aus Einkünften in Bitcoin dieselben steuerrechtlichen Folgen hervorgehen, wie aus Geschäften mit herkömmlichen Geld und wie Einkünfte aus Bitcoin-Mining versteuert werden müssen, wobei Wert aus nichts geschaffen wird.
Würde Bitcoin wie eine Ware behandelt werden, würden Steurn nur beim Wechseln in bestehende Währungen fällig (Mehrwertsteuer); aus Käufen mit Bitcoin würden rechtlich Täuschgeschäfte werden.
Das Mining ist in diesem Fall ein normaler Wertschöpfungsprozess.
Bekommt Bitcoin stattdessen den rechtlichen Status eine Währung, so würde bei Transaktionen dieselben Steuern fällig, als hätte man eine bestehende Währung verwendet; Geldumtausch wäre steuerfrei.
Hierbei ist das Versteuern von Bitcoin Mining ein ungelöstes Problem, da es neues Geld in Umlauf bringt, ein Prozess, der vorher Zentralbanken vorbehalten war.
\Cf[34]{schilling}
