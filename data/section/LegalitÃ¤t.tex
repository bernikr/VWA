\subsection{Legalität}

Doch selbst wenn Bitcoin wie eine Währung verwendet werden kann, ist noch immer nicht klar ob es gesetzlich wie eine Währung oder eine Ware behandelt werden soll.
Denn Bitcoin weist sehr ähnliche Eigenschaften wie Gold auf (Knappheit und Fälschungssicherheit), welches allerdings nicht als Währung gilt.
\Cf[32-33]{schilling}
Derzeit befindet sich Bitcoin in einem legalen Graubereich.
Dadurch dass viele Regierungen und Strafverfolgungsbehörden längere Zeit nicht darauf aufmerksam geworden sind, wurden Bitcoins öfters für Delikte wie Geldwäsche \cf[33]{schilling} oder Drogenhandel \cf[46]{peicu} verwendet, was durch die Pseudonymität%
\footnote{\emph{Pseudonymität} bedeutet, dass Bitcoin-Adressen sich zwar nicht direkt auf eine reale Person zurückführen lassen, Bitcoin aber nicht \emph{anonym} ist, da es öffentlich einsehbar ist, wie Bitcoins von Adresse zu Adresse fließen.
Daher kann man Bitcoins bis zu einer bekannten Adresse zurückverfolgen.
Meistens findet man so den Exchange bei dem diese Bitcoins zuletzt gekauft wurden, welcher über die Identität seiner Käufer bescheid wissen sollte.}
von Bitcoin ebenfalls begünstigt wurde.

Besonders der Fall von \emph{Silk Road}, einer Ebay-artigen Website, auf welcher unter anderem Drogen und Waffen gegen Bitcoins angeboten wurden, erlangte 2013 Medienaufmerksamkeit, als das FBI eine Razzia durchführte, die Website vom Netz nahm und 26.000 Bitcoins (damals 4 Millionen US-Dollar wert) beschlagnahmte.
Obwohl die Aktivität auf Silk Road nur einen kleinen Teil der Transaktionen im Bitcoin-Netzwerk ausmachte, erzeugten die Medien den Eindruck es wäre der Zweck von Bitcoin illegale Zahlungen zu tätigen, was auch aufgrund der Tatsache, dass Bitcoin damit zum ersten Mal in die Massenmedien gekommen ist, dem Image der Kryptowährung massiv schadete.
\Cf[46-47]{peicu}

Doch selbst wenn man legalen Geschäften mit Bitcoin nachgehen möchte, bewegt man sich oft in einem Graubereich.
Es ist unklar, ob aus Transaktionen mit Bitcoin dieselben steuerrechtlichen Folgen hervorgehen wie aus Geschäften mit herkömmlichen Geld und wie Einkünfte aus Bitcoin-Mining versteuert werden müssen, bei welchen Wert aus nichts geschaffen wird.

Würde Bitcoin wie eine Ware behandelt werden, würden Steuern nur beim Wechseln in bestehende Währungen fällig (Mehrwertsteuer); aus Käufen mit Bitcoin würden rechtlich Tauschgeschäfte werden.
Das Mining ist in diesem Fall ein normaler Wertschöpfungsprozess.
Bekommt Bitcoin stattdessen den rechtlichen Status einer Währung, so würden bei Transaktionen dieselben Steuern fällig, als hätte man eine bestehende Währung verwendet; Umtausch zwischen Bitocin und anderen Währungen wäre steuerfrei.
Hierbei ist das Versteuern von Bitcoin Mining ein ungelöstes Problem, da es neues Geld in Umlauf bringt:
Ein Prozess, der vorher Zentralbanken vorbehalten war.
\Cf[34]{schilling}
