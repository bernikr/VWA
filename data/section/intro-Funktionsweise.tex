Die in Abschnitt \ref{sec:bitcoinintro} zusammengefassten Funktionen des Bitcoin-Netzwerkes haben viele weitere Aspekte.
Die Blockchain muss als dezentrale Datenbank beispielsweise sicher gegen jede Art von Betrugsversuchen sein und darf daher von niemandem selbstständig verändert werden können.
Auch muss sich beim Durchführen von Transaktionen der Sender authentifizieren können, damit man nur die eigenen Bitcoins ausgeben kann.

Im herkömmlichen Bankwesen sind solche Probleme inexistent oder einfach zu lösen, da die Bank als zentrale Stelle das alleinige Recht hat, die Kontostände zu verändern und die Identität der Kontoinhabenden zu überprüfen.
Um diese Funktionen dezentral durchzuführen sind aufwändige kryptografische Protokolle notwendig.
\cf[2]{nakamoto}
