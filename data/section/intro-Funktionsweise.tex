Die in Abschnitt \ref{sec:bitcoinintro} zusammengefassten Funktionen des Bitcoin-Netzwerkes sind natürlich viel komplexer.
Die Blockchain muss als dezenrale Datenbank beispeilsweise sicher gegen jede Art von Betrugsversuchen sein und darf daher von niemandem selbstständig verändert werden.
Auch muss sich beim Durchführen von Transaktionen der Sender authentifizieren können, damit man nur die eigenen Bitcoins ausgeben kann.

Im herkömlichen Bankwesen sind solche Probleme inexistent oder einfach zu lösen, da die Bank als zentrale Stelle das allinige Recht hat, die Kontostände zu verändern und die Identität der Kontoinhaber zu überprüfen.
Um diese Funktionen dezentral durchzuführen sind aufwändige kryptographische Protokolle notwendig.
\cf[2]{nakamoto}
