\section{Kryptografische Prozesse}

Da Bitcoin ohne Bank auskommt, müssen die dahinterliegenden Prozesse trotz absoluter Transparenz ein hohes Maß an Sicherheit gewähren.
Hierzu werden vorallem zwei kryptografische Prozesse exessiv genutzt:
\begin{description}
    \item[Asymetrische Verschlüsselung], welche benötigt werden um Transaktionen signieren zu können und
    \item[Hashfunktionen], die ein Wesentlicher Bestandteil der Blockchain sind.
\end{description}

\subsection{Asymetrische Verschlüsselung}

Unter Asymetrischer Verschlüsselung werden Verschlüsselungen verstanden, welche unterschiedliche Schlüssel zu Ver- und Entschlüsselung benötigen.
Dies wird meist bewältigt, indem ein Körper definiert wird, in welchem das Inverse zu einem gegebenen Wert nicht trivial zu berechnen ist.
Man generiert daruaf hin ein Schlüsselpaar mithlfe eines Startwertes und hilfsfunktionen, sodass die Ergebnisse inverse sind.

Hat man so ein Schlüsselpaar bezeichnet man geläufigerweise einen davon als privaten und den anderen als öffentlichen Schlüssel.
Nachdem der Ersteller des SChlüsselpaares den öffentlichen Schüssel veröfentlicht hat sind mithilfe von asymetrischer Verschlüsselung sind nun folgende zwei Prozesse möglich:
\begin{enumerate}
    \item Jemand verschlüsselt eine Nachricht mit dem öffentlichen Schlüssel und sendet sie an den Ersteller.
    Entschlüsselbar ist die Nachricht nur mit dem privaten Schlüssel.
    Der vorteil gegenüber symetrischer Verschlüsselung liegt darin, dass es keinen geheimen Schlüsselaustausch gibt, bei dem der Schlüssel abgefangen werden kann.
    \item Der Inhaber privaten Schlüssels verschlüsselt eine Nachricht (oder einen Hash davon) mit dem privaten Schlüssel.
    Jeder kann diese Nachricht nun mit dem öffentlichen Schlüssel entschlüsseln und lesen.
    Da aber nur eine Person den privaten Schlüssel inne hat, lässt sich somit die Authentizität der Nachricht überprüfen.
\end{enumerate}

Die Anwendung bei Bitcoin erfolgt dabei so, dass jeder, der Bitcoins verwenden möchte ein Schlüsselpaar erstellt.
Der öffentlich Schlüssel wird zur Bitcoinadresse.
Möchte man eine Transaktion durchfüren, wird diese mit dem privaten Schlüssel signiert.
Daher ist von jedem überprüfbar, ob eine Transaktion auch wirklich von einer Berechtigten Person durchgeführt wurde.

\subsection{Hashfunktion}

Eine Hashfunktion bildet eine Prüfsumme aus einer Nachricht. \Skpcf{schmeh_kryptografie:_2013}{226}
Es wird unterschieden zwischen kryptografischen und nicht-kryptografischen Hashfunktionen.
Eine nicht-kryptografische Hashfunktion liefert eine Prüfsumme, deren Änderung bei Änderung des Urbildes vorhersehbar ist.
So wäre die Quersumme die einfachste nicht-kryptografische Hashfunktion.

\emph{Kryptographische Hashfunktionen} zeichnen sich dadurch aus, dass man bei gegebenen Hashwert nicht einfach ein Urbild rekonstruieren kann, das diesen Hashwert erfüllt.
