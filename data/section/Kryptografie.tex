\subsection{Kryptografische Prozesse}

Da Bitcoin ohne vertrauenswürdige Einrichtungen auskommt, müssen die dahinterliegenden Prozesse trotz absoluter Transparenz ein hohes Maß an Sicherheit gewährleisten.
Hierzu werden hauptsächlich zwei kryptografische Prozesse exzessiv genutzt:
\begin{description}
    \item[Asymmetrische Verschlüsselung] wird benötigt, um Transaktionen signieren zu können, also um zu beweisen, dass der Ersteller einer Transaktion tatsächlich über die ausgegebenen Bitcoins verfügt.
    \item[Hashfunktionen] dienen dazu, vorhergehende Daten eindeutig zu referenzieren und diese gegen Veränderungen abzusichern.
\end{description}

\subsubsection{Asymmetrische Verschlüsselung}

Bei traditionellen symmetrischen Verschlüsselungen teilen sich Sender und Empfänger einen geheimen Schlüssel, der zum Ver- und Entschlüsseln verwendet wird.
Angenommen Alice möchte eine Nachricht an Bob senden ohne das Eve deren Inhalt mitbekommt, obwohl diese den gesamten Nachrichtenverkehr zwischen den beiden mitliest.
Wenn die beiden über einen gemeinsamen, Eve nicht bekannten Schlüssel verfügen, kann Alice ihre Nachricht verschlüsseln, an Bob senden und dieser kann diese mit demselben Schlüssel entschlüsseln, ohne das Eve den Inhalt erfährt.
Doch wenn sie sich noch nicht auf einen gemeinsamen Schlüssel geeinigt haben, ist es schwer sich diesen trotz Eves Überwachung gegenseitig zukommen zu lassen.

An dieser Stelle setzen \emph{asymmetrische Verschlüsselungen} an.
Unter diesen werden Verschlüsselungen verstanden, welche im Gegensatz zu symmetrischen Verschlüsselungen unterschiedliche Schlüssel zur Ver- und Entschlüsselung benötigen.
Dies wird meist bewältigt, indem eine Gruppe definiert wird, in welcher das Inverse zu einem gegebenen Wert nicht trivial zu berechnen ist.

\paragraph{Mathematische Grundlagen}

Eine \emph{Gruppe} $(G,*)$ bezeichnet eine Menge $G$ von Elementen, auf der eine assoziative Verknüpfung $*$ definiert ist, welche je zwei Elementen ein drittes Element zuordnet.
Außerdem muss es ein neutrales Element $e \in G$ geben, sodass für $\forall x \in G, \quad e*x=x$ gilt, und für jedes Element $x \in G$ ein Inverses $x' \in G$ existiert, sodass $x*x'=e$ gilt.
\Cf[158]{schmeh2007}
Beispiele für Gruppen sind die Menge der ganzen Zahlen mit der üblichen Addition $(\mathbb{Z}, +)$, die reellen Zahlen mit der Multiplikation $(\mathbb{R}, \cdot)$, sowie ein beliebiger Restklassenring mit der Moduloaddition%
\footnote{Modulooperationen sind ähnlich den Operationen in $\mathbb{Z}$, wobei bloß Divisionsreste betrachtet werden.
Zwei Zahlen heißen kongruent ($\equiv$) bezüglich einem Modul $m$, wenn sie bei der Division durch $m$ den selben Rest aufweisen.
So gilt beispielsweise: $1 \equiv 3 \pmod{2}$}
$(\mathbb{Z}_n, +)$.

In den meisten Gruppen ist das Inverse trivial zu berechnen und es existiert eine brauchbare Umkehrfunktion der Verknüpfung.
Betrachten wir beispielsweise eine Gruppe über $\mathbb{R}$ mit der Verknüpfung $c = a \cdot b$.
Hier sind die Umkehrfunktionen $a = \frac{c}{b}$ und $b = \frac{c}{a}$ und das Inverse $x'=\frac{1}{x}$, weil $x \cdot \frac{1}{x} = 1$ gilt und $1$ das neutrale Element bezüglich der Multiplikation ist.

Anders sieht es mit der Modulomultiplikation in der Menge $\{ 1, \dotsc, pq \mid p,q \in \mathbb{P}\}$ aus.
Dabei handelt es sich analog zur Multiplikation um die Verknüpfung $c \equiv a \cdot b \pmod{m}$ in einer Menge an Restklassen.%
\footnote{Hierbei handelt es sich zwar um keine Gruppe, es existiert aber ein Inverses für alle zum Modul teilerfremden Zahlen und ein neutrales Element. \Cf[155]{schmeh2007}}
Die Umkehrfunktion (das Dividieren) ist hierbei nicht möglich und das Inverse ist nur bei Kenntnis von $\varphi(m)$ zu berechnen.
Die Eulersche Phi-Funktion $\varphi(x)$ gibt die Anzahl der zu $x$ teilerfremden Zahlen an, die kleiner als $x$ sind.
Diese Funktion ist nur mithilfe einer Primfaktorzerlegung berechenbar,%
\footnote{Kennt man die Primfaktorzerlegung $p_1 \cdot p_2 \dotsm p_n$ einer Zahl $a$, so gilt $\varphi (a) = (p_1-1) (p_2-1) \dotsm (p_n-1)$.}
welche wiederum für das Produkt zweier hoher Primzahlen nicht in absehbarer Zeit berechenbar ist.
Wählt man also zwei hohe Primzahlen $p$ und $q$ und einen Schlüssel $x$, welcher zu $pq$ teilerfremd ist, und mutipliziert die Primzahlen zu dem Modul $m$, ist das Inverse des Schlüssels nur berechenbar, wenn man die ursprünglichen Primzahlen kennt.
\Cf[166]{schmeh2007}

Betrachten wir nun die Operation $b \equiv a^x \pmod{m}$.
Hierbei ist die Umkehrfunktion --~der diskrete Logarithmus%
\footnote{Der \emph{diskrete} Logarithmus sucht eine \emph{ganze Zahl} $x$, für die die Gleichung $y \equiv a^x \pmod{m}$ bei gegebenem $y$, $a$ und $m$ gilt.}
beziehungsweise die Modulowurzel~-- für größere Werte mit heutigen Algorithmen ebenfalls nicht in absehbarer Zeit zu berechnen.
Kennt man jedoch das Inverse von $x$ so lässt sich aufgrund der Beziehung $a \equiv (a^x)^{x'} \pmod{m}$ mithilfe von $x'$, $b$ und $m$ wieder $a$ ausrechnen.
\Cf[160]{schmeh2007}

\paragraph{Anwendung der asymetrischen Verschlüsselung}

Um von der asymmetrischen Verschlüsselung Gebrauch zu machen, wählt Alice zwei hohe Primzahlen und einen geheimen Schlüssel $g$ und berechnet das Inverse $o$ dazu.
Danach berechnet Alice das Produkt $m$ der Primzahlen und verwirft diese.
Den privaten Schlüssel behält Alice für sich, während sie das Inverse und das Produkt an Bob sendet.
Hierbei ist es egal, ob auch andere Personen den Schlüssel erhalten, da dieser öffentlich ist.
Nachdem Bob den öffentlichen Schlüssel von Alice erhalten hat, sind zwei kryptografische Prozesse möglich:
\begin{enumerate}
    \item Bob verschlüsselt eine Nachricht mit dem öffentlichen Schlüssel und sendet sie an Alice.
    Da nur Alice den privaten Schlüssel hat, kann Eve mit der Nachricht nichts anfangen.
    \item Alice verschlüsselt eine Nachricht mit dem privaten Schlüssel.
    Dieser Prozess wird \emph{Signieren} genannt.
    Somit kann die Nachricht von jedem mit Alice öffentlichem Schlüssel entschlüsselt werden.
    Da aber nur Alice den privaten Schlüssel innehat, lässt sich somit die Authentizität der Nachricht verifizieren.
    So kann Bob sich sicher sein, dass diese nicht auf dem Weg verändert wurde und sicher von Alice stammt.
\end{enumerate}

Die Prozesse der Ver- und Entschlüsselung sind (bis auf den verwendeten Schlüssel) identisch und bestehen aus einer Anwendung der Exponentialfunktion.
Die Verschlüsselung eines Klartextes $K$ zu einem Geheimtext $G$ erfolgt mit dem öffentlichen Schlüssel $o$ über den Modul $m$ mithilfe von $G \equiv K^o \pmod{m}$.%
\footnote{Hierzu muss die Nachricht, welche normalerweise ein Text ist, zuerst in eine Zahl (oder mehrere Zahlen) umgewandelt werden.
Dies ist z.\,B. möglich, indem man die binäre Darstellung eines Textes in das Dezimalsystem umwandelt.}
Die Entschlüsselung erfolgt analog dazu mit dem geheimen Schlüssel $g$: $K \equiv G^g \pmod{m}$.
Dies funktioniert da $g \cdot o \equiv 1 \pmod{m}$ gilt.

Die Anwendung bei Bitcoin erfolgt dabei so, dass jeder, der Bitcoins verwenden möchte ein Schlüsselpaar erstellt.
In diesem Kontext wird der öffentliche Schlüssel Bitcoin-Adresse genannt.
Möchte man eine Transaktion durchführen, wird diese mit dem privaten Schlüssel signiert.
Daher ist von jedem überprüfbar, ob eine Transaktion auch wirklich von der berechtigten Person durchgeführt wurde.

\subsubsection{Hashfunktionen}

Eine Hashfunktion bildet eine Prüfsumme aus einer Nachricht.
Diese kann man benutzen, um festzustellen, ob die Nachricht verändert wurde.
\Cf[200]{schmeh2007}
Es wird zwischen kryptografischen und nicht"=kryptografischen Hashfunktionen unterschieden.
Eine \emph{nicht-kryptografische} Hashfunktion liefert eine Prüfsumme (Hash) einer Nachricht (Urbild), deren Änderung bei Änderung des Urbildes vorhersehbar ist.
So ist die Quersumme eine einfache nicht-kryptografische Hashfunktion.
Nicht-kryptografische Hashfunktionen findet man zum Beispiel als Prüfziffern in Kontonummern, ISBNs und Barcodes vor, wo sie ein irrtümliches Ändern der Nachricht (in diesem Fall der Nummer) verhindern sollen.
Vertippt man sich zum Beispiel bei Eingabe einer Kontonummer, so kann der Computer feststellen, dass die eingegebene Nummer nicht zur angegebenen Prüfziffer passt und jene als Fehlerhaft markieren.
Hierbei ist bei einer Änderung des Urbildes vorhersehbar, wie sich der Hash ändern wird.
Verringert man zum Beispiel eine Ziffer um 1, so wird auch die Quersumme um 1 weniger.
Folglich lässt sich zu einer gegebenen Zahl ein Urbild finden, das jene als Prüfsumme hat.
\Cfmulti[35-36]{schneier}[200-201]{schmeh2007}

\emph{Kryptografische Hashfunktionen} zeichnen sich dadurch aus, dass man bei gegebenen Hashwert nicht einfach ein Urbild konstruieren kann, das jenen erfüllt.
Ebenso soll es mit realistischem Aufwand nicht möglich sein, zwei Urbilder zu finden, die denselben Hashwert haben.
\Cf[289-290]{wobst}
Solche Einweg-Hashes%
\footnote{\emph{Einweg} bezieht sich in diesem Kontext darauf, dass nur die Berechnung von Urbild zu Hash möglich ist, man aus dem Hash jedoch keinerlei Informationen über das Urbild schließen kann.}
erlauben es zum Beispiel Prüfsummen zu erstellen, die nicht nur gegen zufällige Veränderungen Schutz bieten, sondern auch gegenüber böswilligen Angriffen.
Übermittelt man jemanden zum Beispiel eine ausführbare Datei und teilt ihm auf einem sicheren Weg den Hash der Datei mit, kann sich der Empfänger --~sofern die Hashwerte übereinstimmen~-- sicher sein, dass die Datei nicht von Dritten manipuliert wurde.
Denn wenn ein Angreifer zum Beispiel versucht einen Virus in die Datei einzuschleusen, wird sich der Hashwert verändern, egal wie sehr sich der Angreifer um Geheimhaltung bemüht.

Möchte man eine Nachricht signieren, so hat es oft Vorteile nur den Hash der Nachricht zu signieren und der Nachricht beizulegen, anstatt die gesamte Nachricht mit dem privaten Schlüssel zu verschlüsseln, da asymetrische Verschlüsselungen oft nicht für große Datenmengen geeignet sind.
