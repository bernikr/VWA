Auch wenn Bitcoin vom Konzept wie eine ideale Währung wirken kann, gibt es dennoch einige Probleme, die es davon abhalten weiter verbreitet zu sein.
Denn Bitcoin ist bei weitem noch nicht reif genug, um es mit dem Euro oder dem Dollar auf eine Stufe zu stellen, müssen doch erst einige konzeptuelle und technische Probleme aus dem Weg geräumt werden.

\section{Instabilität}

Wie schon in Abschnitt \ref{sec:speculation} erwähnt gibt es einen Teufelskreis der Instabilität.
Damit Bitcoin als wirkliche Währung verwendet werden kann und nicht nur als Zahlungsmittel (Der Unterschied besteht darin, dass Shops die Preise derzeit in Euro oder Dollar anschreiben und nach dem aktuellen Bitcoinkurs umrechnen) muss dieser Kreis durchbrochen werden.
Bitcoin funktioniert derzeit zwar sehr gut als dezentrales Zahlungsmittel und als Geldanlage, doch kaum jemand würde auf die Idee kommen, nur noch Bitcoin zu verwenden und herkömmlichen Währungen den Rücken zuzukehren.

\section{Praktikabilität im Alltag}

Während Bitcoin-Befürworter der Meinung sind, dass es notwendig ist, dass Transaktionen nicht rückgängig gemacht werden können, bringt das bei der Verwendung von Bitcoin Risiken mit sich.
Bestellt man beispielsweise etwas bei Onlineshop eines böswilligen Betreibers mithilfe eine Kreditkarte und die bestellten Waren kommen nicht an, so reicht meistens ein Anruf bei dem Kreditkartenunternehmen um den Betrag zurückbuchen zu lassen.
Hätte man hingegen mit Bitcoins bezahlt so hat man bei einem unkooperativen Händler keine Möglichkeit sein Geld zurückzubekommen.

Außerdem gibt es für die Mehrheit der Benutzer keinen Grund Bitcoin zu verwenden, da bisherige Zahlungsmittel oft einfacher zu verwenden sind.
Die meisten Gründe die für Bitcoin sprechen (dezentral, global, etc.) sind ideologischer Natur.
Für den Normalverbraucher ist es schlicht umständlich sein Gehalt in Euro zu erhalten, sich um einen Teil davon Bitcoins zu kaufen, diese bei einem Shop auszugeben, wo er auch mit Euros hätte bezahlen können und welcher diese Bitcoins danach ebenfalls wieder zurück in Euros tauscht.
Bitcoins fügen diesen Transaktionen nur sinnlose Zwischenschritte hinzu die unpraktisch sind und das Verlustrisiko des Geldes erhöhen.
\Cf{simonite}

\section{Betrug}

Durch die Pseudonymität und Unumkehrbarkeit von Bitcointransaktionen wird Bitcoin des Öfteren für betrügerische Zwecke eingesetzt.
So sind Bitcoin das bevorzugte Zahlungsmittel von Erpresserviren und böswillige Shops akzeptieren Bitcoins ohne jemals Waren auszusenden.
Auch der private Schlüssel auf der eigenen Festplatte bietet Hackern eine Angriffsmöglichkeit.
Gerade bei größeren Unternehmen der Bitcoin-Welt kann ein Bitcoin Diebstahl fatal sein.
So wurden Bitpay 1,8 Millionen US-Dollar entwendet, indem es dem Dieb gelungen war, \begin{quote}{simonite}den Firmenchef zum Überweisen der Summe zu bewegen, in dem er per E-Mail den Finanzchef nachahmte.\end{quote}

\section{Nachhaltigkeit}

Laut einem Artikel von Christopher Malmo aus dem Online-Magazin Motherboard \begin{quote}{malmo}benötigt eine einzige Bitcoin-Transaktion so viel Strom, dass man damit 1,57 US-Haushalte mit Energie versorgen könnte -- und das einen ganzen Tag lang.\end{quote}
Denn das Bitcoin-Mining ist ein energieaufwändiger Prozess und durch den freien Wettbewerb des Mining wird die verwendete Rechenleistung und damit der verbrauchte Strom solange steigen, wie es profitabel ist.

Um den Stromverbrauch zu senken müsste entweder die Mininghardware effizienter werden (was allerdings wieder die Anzahl der Miner vergrößert) oder das Mining weniger profitabel.
Doch versucht man die Belohnung eines Blockes durch ändern des Bitcoin-Codes zu verringern, wird dieser Fork vermutlich nie angewandt werden, da letztendlich die Miner diejenigen sind, die mit ihrer Verwendung von Abwandlungen der Originalsoftware über Änderungen in der Bitcoin-Software abstimmen.
Man kann nur darauf hoffen, dass die in Abschnitt \ref{subsec:mining} beschriebene Halbierung der Belohnung eines Blockes alle vier Jahre genügend Miner aufhören lässt und so den Stromverbrauch senkt.

\section{Block-Size-Debatte}
\label{sec:blocksize}

Doch nicht alle Probleme sind vom Konzept so grundlegenden und scheinbar unlösbar.
2010 wurde die Größe eines Blockes von 36~MB auf 1~MB herabgesetzt, um zu vermeiden, das Miner Spam-Blöcke mit mehreren Megabyte an sinnlosem Inhalt erstellen und diese die Blockchain unnötig vergrößern.
Damals war das auch kein Problem, da sich die durchschnittliche Größe eines Blockes im Bereich von wenigen Kilobyte befand.
Anfang 2015 stellte man allerdings fest, dass aufgrund der immer weiteren Verbreitung von Bitcoin die Anzahl der Transaktionen und somit auch die Größe der Blöcke rasant anstieg und drohte an die 1~MB-Grenze zu gelangen.
\Cf{caffyn}

Die Folgen des Erreichens wären nach einigen Meinungen fatal.
Da nicht mehr alle Transaktionen in die Blöcke aufgenommen werden können, kommt es zu einem Wettbewerb bei der Transaktionsgebühr.
Nur noch Transaktionen welche den Minern die meisten Bitcoins überlassen werden bestätigt, andere Transaktionen mit zu wenige Gebühr befinden sich dann für immer im Rückstau der Miner.
Während viele dabei das Ende von Bitcoin befürchten, sind andere der Meinung, dass diese Form des freien Marktes die Lösung der Debatte sei.

Die Vielzahl der anderen Lösungsvorschläge haben den Punkt, das Limit anzuheben gemein, unterscheiden sich doch in Details wie, der Höhe des Limits oder den Verlauf über längere Zeit.
Unter anderem schlägt Gavin Anderson, der ehemalige leitende Entwickler des Bitcoin Projektes vor, das Limit sofort auf 8~MB anzuheben und danach alle zwei Jahre eine Vergrößerung um 40\% durchzuführen.
Andere Vorschläge sind zum Beispiel die sofortige Anhebung auf 2~MB und danach eine jährliche Erhöhung von 17,7\%.
\Cf{caffyn}

Während diese Details unwichtig erscheinen, ist es aufgrund der dezentralen Natur des Bitcoin-Netzwerkes nicht möglich, Änderung vorzunehmen, solange kein Konsens erreicht ist.
Erst wenn sich die Entwickler einig sind, kann einer der Vorschläge umgesetzt werden.
Doch nicht nur die Entwickler haben ein Wort mitzureden:
Wenn nach Veröffentlichung einer neuen Version mehr als die Hälfte der Miner bei der alten Version bleibt, wird die längste Blockchain weiterhin die alte Version verwenden und es wird zu keinen Änderungen kommen.
Die Zustimmung der Miner zu bekommen ist jedoch auch schwierig, sind doch mehr als 50\% der Miner in China angesiedelt wo die durchschnittliche Internetgeschwindigkeit mit 3,4 Megabyte pro Sekunde \cf{china} relativ langsam ist, weshalb sie befürchten, dass sie, sollte das Limit erhöht werden, neue Blocke erst zu spät erhalten und dadurch einen Nachteil gegenüber Ländern mit schnellerer Internetverbindung haben.
\Cf{caffyn}

Der letze große Vorschlag in der Debatte wird \emph{Segregated Witness} genannt.
Im Gegensatz zu den anderen Vorschlägen, bei welchen es sich um \emph{Hard-Forks} handelt (d.\,h. die neue Version beinhaltet nicht rückwärtskompatible Änderung, wodurch Bitcoin in eine neue und eine alte Version aufgespalten werden) ist \emph{Segrageted Witness} ein \emph{Soft-Fork}, bei welchem alte Bitcoin-Clients weiterhin verwendbar sind und nur die Miner auf die neueste Version updaten müssen.
\emph{Segrageted Witness} schlägt vor, die eigentlich Transaktion ("`Wer schickt wie viele Bitcoins an wen?"') von dem Teil zu trennen der nur für Zeugen von Relevanz ist (v.a. Signatur) und diesen Teil nur solange zu speichern, solange er von Relevanz ist und ihn daher nicht in die Blockchain aufnehmen.
Denn sobald eine neue Transaktion auf eine alte aufbaut, ist es nicht mehr nötig, die alte Transaktion auf Gültigkeit zu überprüfen, da diese sowieso nicht mehr verändert werden kann.
Da diese Informationen für Zeugen derzeit zirka zwei Drittel der Blockchain beanspruchen, könnte dadurch viel Speicherplatz gespart werden und das 1~MB-Limit aufrechterhalten werden.
\Cf{rizzo}

Doch es gibt auch Kritik an diesem Vorschlag:
Aufgrund der Bedingung rückwärtskompatibel zu sein, muss die Signatur in einem bereits bestehenden Feld der Transaktion (im konkreten Fall das \emph{Coinbase}-Feld) eingeschlossen werden, welches dadurch zweckentfremdet wird.
Da Miner dieses Feld unter anderem dafür verwenden über Änderungen in der Bitcoin Software abzustimmen, gibt es Befürchtungen, dass durch \emph{Segregated Witness} die Mining-Software verkomplizieren, da dieses Feld nun abhängig vom Rest der Transaktionen ist.
\Cf{rizzo}

Letztendlich ist die \emph{Block-Size-Debatte} noch lange nicht abgeschlossen, doch es gibt Hoffnung zu einem Ende zu kommen, wie auch Young in seinem Artikel auf \url{newsbtc.com} im Jänner 2016 darlegt:

\begin{longquote}{young}
Due to the disputes between these two sides, miners, bitcoin organizations, startups, core developers and experts failed to reach a consensus, extending the debate for years.
However, instant cryptocurrencies exchange Shapeshift.io founder and CEO Erik Voorhees predict that the bitcoin community wil reach a consensus in 2016.
\end{longquote}

Da die durchschnittliche Größe eines Blockes derzeit erst etwas über einem halben Megabyte liegt, ist diese Debatte zwar wichtig für die Zukunft von Bitcoin, doch nicht dringend für die aktuelle Situation, da derzeit noch deutlich Spielraum nach oben vorhanden ist, wodurch Bitcoin zumindest in naher Zukunft noch voll funktionsfähig bleiben sollte.
\Cf{young}
