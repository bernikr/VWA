\section{Generelle Probleme}

Doch Bitcoin ist bei weitem noch nicht reif genug, um als Währung weit verbreitet zu sein, müssen doch erst einige konzeptuelle und technische Probleme aus dem Weg geräumt werden.

\subsection{Instabilität}

Wie schon in Abschnitt \ref{sec:speculation} erwähnt gibt es einen Teufelskreis der Instabilität.
Damit Bitcoin als wirkliche Währung verwendet werden kann und nicht nur als Zahlungsmittel (Der Unterschied besteht darin, dass Shops die Preise derzeit in Euro oder Dollar anschreiben und nach dem aktuellen Bitcoinkurs umrechnen) muss dieser Kreis durchbrochen werden.
Bitcoin funktioniert derzeit zwar sehr gut als dezentrales Zahlungsmittel und als Geldanlage, doch kaum jemand würde auf die Idee kommen, nur noch Bitcoin zu verwenden und herkömmlichen Währungen den Rücken zuzukehren.

\subsection{Praktikabilität im Alltag}

Währden Bitcoin-Befürworter der Meinung sind, dass es notwendig ist, dass Transaktionen nicht rückgängig gemacht werden können, bringt das bei der Verwendung von Bitcoin Risiken mit sich.
Bestellt man Beispielsweise etwas bei einem böswilligen Onlineshop mithilfe eine Kreditkarte und die Bestellten waren kommen nicht an, so reicht meistens ein Anruf bei dem Kreditkartenunternehmen um den Betrag zurückbuchen zu lassen.
Hätte man hingegen mit Bitcoins bezahlt so hat man bei einem unkooperativen Händler keine Möglichkeit sein Geld zurückzubekommen.

Außerdem gibt es für die Mehrheit der Benutzer keinen Grund Bitcoin zu verwenden, da bisherige Zahlungsmittel oft einfacher zu verwenden sind.
Die meisten Gründe die für Bitcoin sprechen (dezentral, global, etc.) sind ideologischer Natur.
Für den Normalverbraucher ist es schlicht umständlich sein Gehalt in Euro zu erhalten, sich um einen Teil davon Bitcoins zu kaufen, diese bei einem Shop auszugeben, wo er auch mit Euros hätte bezahlen können und welcher diese Bitcoins danach ebenfalls wieder zurück in Euros tauscht.
Bitcoins fügen diesen Transaktionen nur sinnlose Zwischenschritte hinzu die unpraktisch sind und das Verlustrisiko des Geldes erhöhen.
\Cf{simonite}

\subsection{Betrug}

Durch die Pseudonymität und Unumkehrbarkeit von Bitcointransaktionen wird Bitcoin des öfteren für betrügerische Zwecke eingesetzt.
So sind Bitcoin das bevorzugte Zahlungsmittel von Erpresserviren und böswillige Shops akzepieren Bitcoins ohne jemals Waren auszusenden.
Auch der private Key auf der eigenen Festplatte bietet Hackern eine Angriffsmöglichkeit.
Gerade bei größeren Unternehmen der Bitcoin-Welt kann ein Bitcoin Diebstahl fatal sein.
So wurden Bitpay 1,8 Millionen US-Dollar entwendet, indem es dem Dieb gelungen war, \begin{quote}{simonite}den Firmenchef zum Überweisen der Summe zu bewegen, in dem er per E-Mail den Finanzchef nachahmte.\end{quote}

\subsection{Nachhaltigkeit}

Laut einem Artikel von Christopher Malmo aus dem Online-Magazin Motherboard \begin{quote}{malmo}benötigt eine einzige Bitcoin-Transaktion so viel Strom, dass man damit 1,57 US-Haushalte mit Energie versorgen könnte -- und das einen ganzen Tag lang.\end{quote}
Denn das Bitcoin-Mining ist ein stromaufwändiger Prozess und freien Wettbewerb des Mining wird die verwendete Rechenleistung und damit der verbrauchte Strom solange steigen, wie es profitabel ist.

Um den Stromverbrauch zu senken müsste entweder die Mininghardware effizienter werden (was allerdings wieder die Anzahl der Miner vergrößert) oder das Mining weniger profitabel.
Doch versucht man die Belohnung eines Blockes durch ändern des Bitcoin-Codes zu veringern, wird dieser Fork vermutlich nie angewandt werden, da letzendlich die Miner diejenigen sind, die mit ihrer Verwendung von Abwandlungen über Änderungen in der Bitcoin-Software bestimmen.
Man kann nur darauf hoffen, dass die in Abschnitt \ref{subsec:mining} beschriebene Halbierung der Belohnung eines Blockes alle vier Jahre genügend Miner aufhören lässt und so den Stromverbrauch senkt.
