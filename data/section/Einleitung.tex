\emph{Bitcoin} verspricht eine Währung zu sein, die es --~unabhängig von zentralen Einrichtungen wie Banken oder Staaten~-- ermöglicht, beliebite Summen an Geld weltweit sofort zu transferieren.
Diese Arbeit beschäftigt sich damit, wie Bitcoin dieses Versprechen halten kann, in dem es die Funktionsweise und Ökonomie von Bitcoin beleuchtet.
Letztendlich befasst sie sich auch mit einigen Problematiken von Bitcoin.
Hierbei können allerdings bei weitem nicht alle Probleme behandelt werden, da diese einerseits den Rahmen der Arbeit sprengen würden und sich die Probleme andererseits bereits in wenigen Wochen komplett verändern können, sofern es laufende Diskussionen darüber gibt.
Gerade die in Abschnitt \ref{sec:blocksize} geschilderte \emph{Block-Size-Debatte} ist ein sehr aktuelles Thema, in welchem sich zwischen Verfassen dieser Arbeit und der Presentation vermutlich einiges ändern wird.

Die Arbeit ist eine reine Literaturarbeit, der technische Teil stützt sich dabei vorallem auf das unter dem Pseudonym \name{Satoshi Nakamoto} veröffentlichtem White Paper "`Bitcoin: A Peer-to-Peer Electronic Cash System"', welches noch vor der Entwicklung von der Bitcoin-Software publiziert wurde und somit die Grundlage der Bitcoin-Technologie bildet. \parencite{nakamoto}
Da nicht alle weiteren Entwicklungen der Software in dem Paper erleutert werden, ist der Artikel "`How the Bitcoin protocol actually works"' von \name{Michael Nielsen}, welcher einen sehr guten Überblick über die technische Funktionsweise bietet, ebenfalls eine der Hauptquellen dieser Arbeit. \parencite{nielsen}
