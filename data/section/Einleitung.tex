\emph{Bitcoin} verspricht eine Währung zu sein, die es --~unabhängig von zentralen Einrichtungen wie Banken oder Staaten~-- ermöglicht, beliebige Summen an Geld weltweit sofort zu transferieren.
Bitcoin stellt sich somit in direkte Konkurrenz zu unserem herkömmlichen Bankwesen, indem es die Macht über das Geld den Banken und Staaten entreißen und in die Hände des Volkes legen möchte.

Die vorliegende Arbeit versucht darzulegen, wie Bitcoin dieses Versprechen halten kann.
Dazu wird zuerst die technische Funktionsweise erklärt, danach wird dargelegt, woher sich der Wert der Bitcoins begründet und ob es als Währung betrachtet werden kann.
Zum Schluss werden noch einige Probleme erläutert, die Bitcoin im Moment hat.
Hierbei können allerdings bei weitem nicht alle Problematiken behandelt werden, da diese einerseits den Rahmen der Arbeit sprengen würden und sich die Probleme andererseits bereits in wenigen Wochen komplett verändern können, sofern es laufende Diskussionen darüber gibt.
Gerade die in Abschnitt \ref{sec:blocksize} geschilderte \emph{Block-Size-Debatte} ist ein sehr aktuelles Thema, in welchem sich zwischen Verfassen dieser Arbeit und der Präsentation vermutlich einiges ändern wird.

Die Arbeit ist eine reine Literaturarbeit, der technische Teil stützt sich dabei vor allem auf das unter dem Pseudonym \name{Satoshi Nakamoto} veröffentlichte White Paper "`Bitcoin: A Peer-to-Peer Electronic Cash System"', welches noch vor der Entwicklung der Bitcoin-Software publiziert wurde und somit die Grundlage der Bitcoin-Technologie bildet. \parencite{nakamoto}
Da nicht alle weiteren Entwicklungen der Software in dem Paper erläutert werden, ist der Artikel "`How the Bitcoin protocol actually works"' von \name{Michael Nielsen}, welcher einen sehr guten Überblick über die technische Funktionsweise bietet, ebenfalls eine der Hauptquellen dieser Arbeit. \parencite{nielsen}

\subsection{Was ist Bitcoin?}
\label{sec:bitcoinintro}

Einerseits ist \emph{Bitcoin} der Name einer digitalen Geldeinheit, andererseits die Bezeichnung des dezentralen Zahlungssystems (auch \emph{Bitcoin-Netzwerk} genannt), das es ermöglicht Bitcoins (die \emph{Geldeinheit}) zu senden.%
\footnote{Im Englischen wird zur Unterscheidung oft Bitcoin als Geldeinheit klein geschrieben und als Zahlungssystem groß; im Deutschen ist diese Unterscheidung leider nicht möglich.}
Das Bitcoin-Netzwerk speichert eine dezentrale (d.\,h. nicht von einer zentralen Stelle abhängige) Datenbank, \emph{Blockchain} genannt, in der alle Transaktionen, also Überweisungen von Bitcoins zwischen Bitcoin-Adressen, vermerkt werden.
Jeder kann die Bitcoin-Software herunterladen, wodurch er Teil des Bitcoin-Netzwerkes wird und die Möglichkeit hat, Bitcoin-Adressen zu erstellen und Bitcoins an andere Adressen zu überweisen.
\cf{nielsen}
