\section{Benutzung}

Während die technische Funktionsweise relativ kompliziert ist, um ein adäquates Level an Sicherheit zu gewährleisten, ist die Benutzung vergleichsweise einfach.
Da inzwischen neben der ursprünglichen Bitcoin Software auch viele Anbieter von online Bitcoin-Geldbörsen existieren, kann die Benutzerfreundlichkeit und Sicherheit je nach Methode stark variieren.

\subsection{Benutzung des ursprünglichen Bitcoin-Clients}

Möchte man die ursprüngliche Software verwenden, kann man diese von \url{http://bitcoin.org/} herunterladen.
Nach der Installation, muss diese allerdings die gesamte Blockchain herunterladen, welche inzwischen 50~GB umfasst und stetig wächst. (Stand: Dezember 2015)
Ist dies abgeschlossen, kann man sich mit einem Klick eine Bitcoin-Adresse erstellen.
Gibt man diese Adresse weiter, ist man bereit Bitcoins zu empfangen.
Zum Senden muss man bloß die Adresse des Empfängers und den gewünschten Betrag in die Software eintragen, und auf Senden klicken.

Allerdings ist zu beachten, dass die Software die privaten Schlüssel in einer Datei auf der Festplatte speichert.
Man sollte daher einerseits darauf aufpassen, dass niemand Zugriff auf diese Datei hat und andererseits, dass man ein Backup davon hat.
Denn ein Verlust dieser Datei bedeutet permanenten Verlust der im Besitz befindlichen Bitcoins.

\subsection{Benutzung eines Online-Dienstes}

Onlinedienste wie Coinbase oder Blockchain.info scheinen all diese Probleme zu beheben.
Zur Erstellung einer Bitcoin-Adresse ist lediglich das Anlegen eines Accounts notwendig, das Senden und Empfangen geht genauso einfach wie bei der ursprünglichen Software und außerdem kann man jederzeit von jedem internetfähigem Gerät auf seine Bitcoins zugreifen.

Der Preis dafür ist jedoch, dass man diesen Diensten praktisch seine Bitcoins überlässt, in der Hoffnung, dass sie ihr versprechen halten und nicht mit den Bitcoins all ihrer User verschwinden.
