Bitcoin bietet dank moderner kryptografischer Verfahren erstmals die Möglichkeit Geld ohne Kontrolle durch Staaten und Banken zu verwenden.
Dies wird ermöglicht, indem jede Transaktion vom Urheber signiert in dem ewigen Logfile der Blockchain gespeichert wird.
Die daraus resultierende Währung ist in ihren Eigenschaften Bargeld oder sogar Gold näher als bisherigen digitalen Zahlungsmitteln, da Bitcointransaktionen irreversibel sind und die in Umlauf gebrachte Menge ein Maximum hat.

Aufgrund dieser Eigenschaften wurde Bitcoin hauptsächlich von Menschen verwendet, welche aus ideologischen Gründen nicht damit einverstanden waren, dass der Staat und Banken Kontrolle über ihr Geld haben.
Auch werden Bitcoins aus demselben Grund teilweise zu kriminellen Zwecken verwendet.
Allerdings findet Bitcoin langsam weitere Akzeptanz, sodass inzwischen auch Großkonzerne wie Microsoft Bitcoins akzeptieren.
Von der breiten Masse verwendet werden wird Bitcoin vermutlich nie, da das Verwenden von Bitcoins gegenüber herkömmlichen Zahlungsmitteln für den Normalverbraucher zu umständlich ist und ein gewisses Risiko mit sich bringt.
Außerdem gibt es in vielen Ländern noch keine gesicherte Rechtslage von Bitcoin, wodurch sich Benutzer oft in einer Grauzone bewegen.

Der Preis einer Bitcoin folgt dem freien Markt und hat dadurch eine hohe Volatilität, was Bitcoin für Spekulanten interessant machen kann.
Mehr Spekulanten sorgen allerdings für mehr Volatilität, wodurch Bitcoin uninteressanter für normale Benutzer wird.
Damit Bitcoin zu einer sinnvollen Währung werden kann, muss dieser Teufelskreis durchbrochen werden, wozu es allerdings keine aktuell diskutierten Vorschläge gibt.

Diese Arbeit versucht nicht neue Erkenntnisse zu gewinnen, sondern sich mit Bitcoin sowohl aus einer informatischen als auch ökonomischen Sicht auseinanderzusetzen, um zu erklären wie Bitcoin funktioniert.
Die aufgekommene Frage, ob Bitcoin eine Zukunft hat oder untergehen wird, kann hier leider nicht behandelt werden, da man sich dazu noch detaillierter mit den aktuellen Problematiken auseinandersetzen müsste.
