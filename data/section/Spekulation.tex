\subsection{Spekulationsgut oder Währung?}
\label{sec:speculation}

Ein anderer Grund warum Kritiker Bitcoin nicht als Währung anerkennen wollen, ist die Tatsache, dass Bitcoin von vielen als Spekulationsgut benutzt wird anstatt als Währung.
Durch die begrenzte Menge an Bitcoins sollte der Preis theoretisch proportional zur Verbreitung von Bitcoin sein -- Wenn mehr Leute Bitcoin verwenden wollen, muss der Preis steigen, damit es jeder wie bisher verwenden kann, da sonst zu wenig Geld zur Verfügung stehen würde.
Diese in Aussicht gestellte Deflation hält Bitcoinbenutzer aber davon ab, mit Bitcoins zu zahlen, da zu erwarten ist, dass diese im Preis steigen werden, und praktisch alle Händler, welche Bitcoin annehmen, auch herkömmliche Währungen akzeptieren.

Des Weiteren ist der Preis sehr instabil, da kein Händler den Nennpreis seines Produktes in Bitcoin angeben wird, sofern seine gesamte Produktion in Euro oder Dollar bezahlt wird.
Daher verwenden die meisten Händler Dienste wie Coinbase oder Bitpay, welche zum Kaufzeitpunkt den Preis in Bitcoins umrechnen, diese vom Käufer einkassieren und dem Händler den entsprechenden Betrag in einer Währung seiner Wahl zukommen lassen.
Diese Flexibilität erlaubt es Bitcoin in kürzester Zeit extreme Kursschwankungen durchzumachen, da der Preis nur von den Börsen abhängig ist und nicht durch andere Transaktionen stabilisiert wird.
Würden Händler beispielsweise die Bitcoinpreise ihrer Produkte händisch anpassen, erzeugt das eine Trägheit des Marktes, wodurch der Kurs nicht so stark schwanken könnte.

Durch diese Instabilität werden außerdem noch Spekulanten angelockt, welche durch häufiges Kaufen und Verkaufen erhoffen, die Schwankung ausnutzen zu können und Profit zu machen.
Doch das verstärkt die Schwankung nur noch stärker, weshalb es unwahrscheinlich ist, das Bitcoin demnächst an die Stabilität bisheriger Währungen herankommt.

Es gab in den vergangenen Jahren mehrere Bitcoin-Blasen, bei welchen der Preis zuerst innerhalb weniger Wochen um das zehn- bis hundertfache stieg und darauf in einigen Tagen um mehr als die Hälfte fiel.
Im Gegensatz zu herkömmlichen Spekulationsblasen war der Bitcoinpreis allerdings nachher immer wesentlich höher als zuvor.
Solche Blasen locken natürlich auch Spekulanten an, welche zum Beispiel mit Leerverkäufen zu deren Entstehung beitragen.
\Cf[89-91]{peicu}
