\section{Spekulationsgut oder Währung?}

Ein anderer Grund warum Kritiker Bitcoin nicht als Währung anerkennen wollen, ist die Tatsache, dass Bitcoin von vielen als Investions oder Spekulationsgut benutzt wird anstatt als Währung.
Durch die begrenzte Menge an Bitcoins sollte der Preis theoretisch indirekt proportional zur Verbreitung von Bitcoin sein -- Wenn mehr Leute Bitcoin verwenden wollen, muss der Preis steigen, damit es jeder wie bisher verwenden kann, da sonst zu wenig Bitcoins zur verfügung stehen würden.
Diese in Aussicht gestellte Deflation hält Bitcoinbenutzer aber davon ab, mit Bitcoin zu zahlen, da zu erwarten ist, dass sie im Preis steigen werden und praktische alle Händler die Bitcoin annehmen auch herkömmliche Währungen akzeptieren werden.

Desweiteren ist der Preis sehr istabil, da kein Händler den Nennpreis seines Produktes in Bitcoin angeben wird, sofern seine gesammte Produktion in Euro oder Dollar bezahlt wird.
Daher verwenden die meisten Händler Dienste wie Coinbase oder Bitpay, welche zum Kaufzeitpunkt den Preis in Bitcoin umrechnen, dieses vom Käufer einkassieren und dem Händler den entsprechenden Betrag in einer Währung seiner Wahl zukommen lassen.
Die flexibilität erlaubt es Bitcoin in kürzerster Zeit extreme Kursschwankungen durchzumachen, da der Preis nur von den Börsen abhängig ist und nicht durch andere Transaktionen stabilisiert wird.
Würden Händler beispielsweise die Bitcoinpreise ihrer Produkte händisch anpassen erzeugt das eine Trägheit des Marktes, wodurch der Kurs nicht so stark schwankt.

Duch diese Insabilität werden außerdem noch Spekulanten angelockt, welche durch häufiges (ver-)kaufen erhoffen die Schwankung auszunutzen und Profit zu machen.
Doch das verstärkt die Schwankung nur noch stärker, weshalb es unwahrscheinlich ist, das Bitcoin demnächst an die Stabilität bisheriger Währungen herankommt.

Es gab in den vergangen Jahren meherere Bitcoin-Blasen, bei welchen der Preis zuerst inerhalb weniger Wochen um das zehn- bis hundertfache steigt und darauf in einigen Tagen um mehr als die Hälfte fällt.
Im Gegensatz zu herkömmlichen Spekulationsblasen war der Bitcoinpreis bisher nacher immer wesentlich höher als vorher.
Solche Blasen locken natürlich auch Spekulanten an, welche zum Beispiel mit Leerverkäufen zu deren Entstehung beitragen.
\Cf[89ff.]{peicu}
