\section{Wert}

Kritiker unterstellen Bitcoin oft, kein Geld zu sein, da es nicht durch Gold gedeckt ist und somit keinen Wert hat.
Doch auch herkömmliche Währungen sind durch nichts anderes gedeckt, als das Vertrauen in Banken und in den Staat.
\Cf[12]{youssef}
Geschichtlich betrachtet kam Geld in gebrauch, um den Tausch zu vereinfachen.
Anstatt Güter direkt zu Tauschen, wurde gegen Dinge getauscht, von denen man sich sicher war, dass man sie weiter tauschen kann.
Um Hoffnung zu haben, dass etwas auch in der Zukunft Wert hat, muss es vorallem unverderblich sein und nicht leicht zu produzieren.
Der Wert von Geld kommt daher davon, dass man darauf vertraut, dass jemand anderer später Waren gegen dieses Geld tauschen möchte.
\Cfmulti[9f.]{youssef}[167f.]{taghizadegan}[63]{peicu}

Folglich haben Bitcoins einen Wert, weil andere Menschen bereit sind damit zu Handeln.
Es gibt Börsen, auf denen Bitcoins in anderen Währungen gehandelt werden, Onlineshops, in denen man unter anderem mit Bitcoins zahlen kann und auch Geschäfte, die neben Bargeld Bitcoins akzeptieren.
Selbst Großkonzerne wie Microsoft oder Dell akzeptieren inzwischen Bitcoins in ihren Onlineshops.
\Cf[23f.]{schilling}
Und diese Unternehmen akzeptieren Bitcoins letztendlich, weil sie darauf vertrauen, dass sie diese wiederum eintauschen können.

Der numerische Wert von Bitcoin als Wechselkurs in andere Währungen hängt --~wie auf jedem freien Markt~-- nur von Anbebot und Nachfrage ab.
Sogennante Bitcoin-Exchanges bieten Online-Plattformen, auf denen Bitcoin Börsenartig gehandelt werden kann.
Der aktuelle Kurs ist bestimmt durch das Orderbuch, in welchem Interressenten vermerken, wie viel Bitcoins sie zu welchem Preis kaufen und verkaufen wollen.
Normalerweise befindet sich ein Spalt zwischen dem höchsten Kaufgebot und dem niedrigsten Abgebot.
Erst wenn ein Käufer bereit ist mehr zu zahlen beziehungsweise ein Verkäufer bereit ist weniger zu erhalten, kommt es zu einer Transaktion.
Der Preis dieser Transaktion ist daraufhin der aktuelle Kurs.
Daher bedeutet der Kurs nicht, dass seine Bitcoin letztendlich auch diesen Wert haben, wenn man sie verkauft.
\Cf[38]{schilling}
