Befürworter sind der Meinung, dass Bitcoin die Währung der Zukunft und vor allem die Währung des Internets sei.
In letzter Zeit schreiben sogar große Zeitungen ab und zu über die digitale Währung.
Diese Arbeit beschäftigt sich mit den Fragen, die meist beim Lesen dieser Artikel unbeantwortet bleiben:
Wie funktioniert Bitcoin und wie kommen Bitcoins zu ihrem Wert?

Die Arbeit beginnt mit einer technischen Beschreibung des \emph{Bitcoin-Netzwerkes} und dem ewigen Logfile der \emph{Blockchain}.
Danach werden einige Aspekte der Ökonomie der Kryptowährung, wie z.\,B. die Bewertung durch den freien Markt, erörtert und mehrere Problematiken aufgegriffen.
Zur Behandlung des Themas stützt sich die Arbeit dabei auf mehrere Arbeiten und Webseiten über Bitcoin, wobei das Paper "`Bitcoin: A Peer-to-Peer Electronic Cash System"' als Hauptquelle dient, da es selbst die Grundlage für Bitcoin bildet.
