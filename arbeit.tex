\documentclass[12pt,titlepage,abstracton]{scrartcl}

%Biblatex & co
% !TEX root = arbeit.tex
\usepackage[bibstyle=authoryear,citestyle=authoryear,firstinits=false, backend=bibtex]{biblatex}
\makeatletter % keine Probleme mit Namen wie "XY von Z"
\AtBeginDocument{\toggletrue{blx@useprefix}}
\AtBeginBibliography{\togglefalse{blx@useprefix}}
\makeatother
\renewcommand{\postnotedelim}{: }
\DeclareFieldFormat{postnote}{#1}
\DeclareFieldFormat[article,phdthesis,other]{title}{\textit{#1}}
\DeclareFieldFormat[article]{journal}{\textnormal{#1}}
\makeatletter

\newcommand{\txtquote}[1]{"`#1"'} %Anführungszeichen für Zitate im Fließtext

\expandafter\def\expandafter\quotation\expandafter{\quotation\setstretch{1}\small\noindent} %einfacher Zeilenabstand und kleinere Schrift für längere Zitate


\addbibresource{data/quellen.bib}

%Formatierung
\usepackage{a4wide}
\usepackage[english,ngerman]{babel} 
\usepackage[utf8]{inputenc}

\usepackage[style=german]{csquotes}

\usepackage{setspace}
\onehalfspacing

\parindent 0pt
\parskip 12pt

\usepackage{graphicx}

\newcommand{\HRule}{\rule{\linewidth}{0.5mm}}

%Mathematik
\usepackage{amsmath}
\usepackage{amssymb}
\usepackage{listings}

\numberwithin{equation}{subsection}


%Sourcecode
\lstset{basicstyle=\footnotesize\ttfamily, float=htb,lineskip={-5.0pt}}

%Verlinktes Inhaltsverzeichnis
\usepackage[colorlinks,
pdfpagelabels,
pdfstartview = FitH,
bookmarksopen = true,
bookmarksnumbered = true,
linkcolor = black,
plainpages = false,
hypertexnames = false,
citecolor = black,
urlcolor = black] {hyperref}

\begin{document}

\pagenumbering{Roman}

% !TEX root = ../arbeit.tex
\title{Bitcoin: Die Zukunft einer digitalen Kryptowährung}
\author{Bernhard \textsc{Kralofsky}}
\date{DATUM}

%TITLEPAGE
%http://de.wikibooks.org/wiki/LaTeX/_Eine_Titelseite_erstellen
\begin{titlepage}
\begin{center}

%Logo
\includegraphics[width=0.15\textwidth]{img/logo}\\[1cm]    
\textsc{\LARGE Sir Karl Popper Schule}\\[1.5cm]

%Titel
\textsc{\Large Vorwissenschaftliche Arbeit}\\[0.5cm]
\HRule \\[0.4cm]
{ \huge \bfseries \@title}\\[0.1cm]
\HRule \\[1cm]

%Author and Supervisor
\begin{minipage}{0.4\textwidth}
\begin{flushleft} \large
\emph{Autor:}\\
\@author
\end{flushleft}
\end{minipage}
\hfill
\begin{minipage}{0.4\textwidth}
\begin{flushright} \large
\emph{Betreuender Lehrer:} \\
Mag.~Bernhard \textsc{Klimbacher}
\end{flushright}
\end{minipage}

% Unterer Teil der Seite
\vfill
{\large \@date}

\end{center}
\end{titlepage}

\renewcommand{\figurename}{Abb.}

%Zusammenfassungen
\begin{minipage}{\linewidth}
\begin{abstract}
\thispagestyle{plain}
\input{data/zusammenfassung}
\end{abstract}
\begin{otherlanguage}{english} 
\begin{abstract}
\selectlanguage{english}
\thispagestyle{plain}
% !TEX root = ../VWA.tex
\begin{abstractpage}
\begin{mlabstract}{ngerman}
%German Abstract
\end{mlabstract}
\end{abstractpage}

\begin{abstractpage}
\begin{mlabstract}{british}
%English Abstract
\end{mlabstract}
\end{abstractpage}

\end{abstract}
\end{otherlanguage}
\end{minipage}
\newpage

%Inhaltsverzeichnis
\tableofcontents
\newpage

%Text
\pagenumbering{arabic}
\input{data/inhalt}
\newpage

%Literaturverzeichniss
\renewcommand{\refname}{Literaturverzeichnis}
\addcontentsline{toc}{section}{Abbildungs- und \refname}
\nocite{*}
\listoffigures
\printbibheading 
\printbibliography[nottype=misc,heading=subbibliography, 
title={Bücher}] 
\printbibliography[type=misc,heading=subbibliography, 
title={Webseiten}] 


\end{document}